% Options for packages loaded elsewhere
\PassOptionsToPackage{unicode}{hyperref}
\PassOptionsToPackage{hyphens}{url}
%
\documentclass[
]{article}
\usepackage{amsmath,amssymb}
\usepackage{lmodern}
\usepackage{iftex}
\ifPDFTeX
  \usepackage[T1]{fontenc}
  \usepackage[utf8]{inputenc}
  \usepackage{textcomp} % provide euro and other symbols
\else % if luatex or xetex
  \usepackage{unicode-math}
  \defaultfontfeatures{Scale=MatchLowercase}
  \defaultfontfeatures[\rmfamily]{Ligatures=TeX,Scale=1}
\fi
% Use upquote if available, for straight quotes in verbatim environments
\IfFileExists{upquote.sty}{\usepackage{upquote}}{}
\IfFileExists{microtype.sty}{% use microtype if available
  \usepackage[]{microtype}
  \UseMicrotypeSet[protrusion]{basicmath} % disable protrusion for tt fonts
}{}
\makeatletter
\@ifundefined{KOMAClassName}{% if non-KOMA class
  \IfFileExists{parskip.sty}{%
    \usepackage{parskip}
  }{% else
    \setlength{\parindent}{0pt}
    \setlength{\parskip}{6pt plus 2pt minus 1pt}}
}{% if KOMA class
  \KOMAoptions{parskip=half}}
\makeatother
\usepackage{xcolor}
\usepackage[margin=1in]{geometry}
\usepackage{color}
\usepackage{fancyvrb}
\newcommand{\VerbBar}{|}
\newcommand{\VERB}{\Verb[commandchars=\\\{\}]}
\DefineVerbatimEnvironment{Highlighting}{Verbatim}{commandchars=\\\{\}}
% Add ',fontsize=\small' for more characters per line
\usepackage{framed}
\definecolor{shadecolor}{RGB}{248,248,248}
\newenvironment{Shaded}{\begin{snugshade}}{\end{snugshade}}
\newcommand{\AlertTok}[1]{\textcolor[rgb]{0.94,0.16,0.16}{#1}}
\newcommand{\AnnotationTok}[1]{\textcolor[rgb]{0.56,0.35,0.01}{\textbf{\textit{#1}}}}
\newcommand{\AttributeTok}[1]{\textcolor[rgb]{0.77,0.63,0.00}{#1}}
\newcommand{\BaseNTok}[1]{\textcolor[rgb]{0.00,0.00,0.81}{#1}}
\newcommand{\BuiltInTok}[1]{#1}
\newcommand{\CharTok}[1]{\textcolor[rgb]{0.31,0.60,0.02}{#1}}
\newcommand{\CommentTok}[1]{\textcolor[rgb]{0.56,0.35,0.01}{\textit{#1}}}
\newcommand{\CommentVarTok}[1]{\textcolor[rgb]{0.56,0.35,0.01}{\textbf{\textit{#1}}}}
\newcommand{\ConstantTok}[1]{\textcolor[rgb]{0.00,0.00,0.00}{#1}}
\newcommand{\ControlFlowTok}[1]{\textcolor[rgb]{0.13,0.29,0.53}{\textbf{#1}}}
\newcommand{\DataTypeTok}[1]{\textcolor[rgb]{0.13,0.29,0.53}{#1}}
\newcommand{\DecValTok}[1]{\textcolor[rgb]{0.00,0.00,0.81}{#1}}
\newcommand{\DocumentationTok}[1]{\textcolor[rgb]{0.56,0.35,0.01}{\textbf{\textit{#1}}}}
\newcommand{\ErrorTok}[1]{\textcolor[rgb]{0.64,0.00,0.00}{\textbf{#1}}}
\newcommand{\ExtensionTok}[1]{#1}
\newcommand{\FloatTok}[1]{\textcolor[rgb]{0.00,0.00,0.81}{#1}}
\newcommand{\FunctionTok}[1]{\textcolor[rgb]{0.00,0.00,0.00}{#1}}
\newcommand{\ImportTok}[1]{#1}
\newcommand{\InformationTok}[1]{\textcolor[rgb]{0.56,0.35,0.01}{\textbf{\textit{#1}}}}
\newcommand{\KeywordTok}[1]{\textcolor[rgb]{0.13,0.29,0.53}{\textbf{#1}}}
\newcommand{\NormalTok}[1]{#1}
\newcommand{\OperatorTok}[1]{\textcolor[rgb]{0.81,0.36,0.00}{\textbf{#1}}}
\newcommand{\OtherTok}[1]{\textcolor[rgb]{0.56,0.35,0.01}{#1}}
\newcommand{\PreprocessorTok}[1]{\textcolor[rgb]{0.56,0.35,0.01}{\textit{#1}}}
\newcommand{\RegionMarkerTok}[1]{#1}
\newcommand{\SpecialCharTok}[1]{\textcolor[rgb]{0.00,0.00,0.00}{#1}}
\newcommand{\SpecialStringTok}[1]{\textcolor[rgb]{0.31,0.60,0.02}{#1}}
\newcommand{\StringTok}[1]{\textcolor[rgb]{0.31,0.60,0.02}{#1}}
\newcommand{\VariableTok}[1]{\textcolor[rgb]{0.00,0.00,0.00}{#1}}
\newcommand{\VerbatimStringTok}[1]{\textcolor[rgb]{0.31,0.60,0.02}{#1}}
\newcommand{\WarningTok}[1]{\textcolor[rgb]{0.56,0.35,0.01}{\textbf{\textit{#1}}}}
\usepackage{longtable,booktabs,array}
\usepackage{calc} % for calculating minipage widths
% Correct order of tables after \paragraph or \subparagraph
\usepackage{etoolbox}
\makeatletter
\patchcmd\longtable{\par}{\if@noskipsec\mbox{}\fi\par}{}{}
\makeatother
% Allow footnotes in longtable head/foot
\IfFileExists{footnotehyper.sty}{\usepackage{footnotehyper}}{\usepackage{footnote}}
\makesavenoteenv{longtable}
\usepackage{graphicx}
\makeatletter
\def\maxwidth{\ifdim\Gin@nat@width>\linewidth\linewidth\else\Gin@nat@width\fi}
\def\maxheight{\ifdim\Gin@nat@height>\textheight\textheight\else\Gin@nat@height\fi}
\makeatother
% Scale images if necessary, so that they will not overflow the page
% margins by default, and it is still possible to overwrite the defaults
% using explicit options in \includegraphics[width, height, ...]{}
\setkeys{Gin}{width=\maxwidth,height=\maxheight,keepaspectratio}
% Set default figure placement to htbp
\makeatletter
\def\fps@figure{htbp}
\makeatother
\setlength{\emergencystretch}{3em} % prevent overfull lines
\providecommand{\tightlist}{%
  \setlength{\itemsep}{0pt}\setlength{\parskip}{0pt}}
\setcounter{secnumdepth}{-\maxdimen} % remove section numbering
\ifLuaTeX
  \usepackage{selnolig}  % disable illegal ligatures
\fi
\IfFileExists{bookmark.sty}{\usepackage{bookmark}}{\usepackage{hyperref}}
\IfFileExists{xurl.sty}{\usepackage{xurl}}{} % add URL line breaks if available
\urlstyle{same} % disable monospaced font for URLs
\hypersetup{
  hidelinks,
  pdfcreator={LaTeX via pandoc}}

\author{}
\date{\vspace{-2.5em}}

\begin{document}

\hypertarget{simple-linear-regression}{%
\section{Simple linear regression}\label{simple-linear-regression}}

\hypertarget{playbill}{%
\subsection{playbill}\label{playbill}}

First we load the data.

\begin{Shaded}
\begin{Highlighting}[]
\NormalTok{playbill }\OtherTok{\textless{}{-}} \FunctionTok{read.csv}\NormalTok{(}\StringTok{\textquotesingle{}C:/Users/tonyg/Desktop/Academic/Grad/HUDM 5126/playbill.csv\textquotesingle{}}\NormalTok{)}
\FunctionTok{library}\NormalTok{(knitr)}
\end{Highlighting}
\end{Shaded}

\begin{verbatim}
## Warning: 程辑包'knitr'是用R版本4.1.3 来建造的
\end{verbatim}

Then we fit a linear model, \(Y=\beta_0 + \beta_1 + e\) and summarize it
in Table @ref(tab:pb-fit1).

\begin{Shaded}
\begin{Highlighting}[]
\NormalTok{pb\_fit1 }\OtherTok{\textless{}{-}} \FunctionTok{lm}\NormalTok{(CurrentWeek }\SpecialCharTok{\textasciitilde{}}\NormalTok{ LastWeek, }\AttributeTok{data =}\NormalTok{ playbill)}
\FunctionTok{kable}\NormalTok{(}\FunctionTok{summary}\NormalTok{(pb\_fit1)}\SpecialCharTok{$}\NormalTok{coef,}
      \AttributeTok{booktabs =} \ConstantTok{TRUE}\NormalTok{,}
      \AttributeTok{caption =} \StringTok{"Coefficients our linear model."}\NormalTok{)}
\end{Highlighting}
\end{Shaded}

\begin{longtable}[]{@{}
  >{\raggedright\arraybackslash}p{(\columnwidth - 8\tabcolsep) * \real{0.1765}}
  >{\raggedleft\arraybackslash}p{(\columnwidth - 8\tabcolsep) * \real{0.1912}}
  >{\raggedleft\arraybackslash}p{(\columnwidth - 8\tabcolsep) * \real{0.1912}}
  >{\raggedleft\arraybackslash}p{(\columnwidth - 8\tabcolsep) * \real{0.1618}}
  >{\raggedleft\arraybackslash}p{(\columnwidth - 8\tabcolsep) * \real{0.2794}}@{}}
\caption{Coefficients our linear model.}\tabularnewline
\toprule()
\begin{minipage}[b]{\linewidth}\raggedright
\end{minipage} & \begin{minipage}[b]{\linewidth}\raggedleft
Estimate
\end{minipage} & \begin{minipage}[b]{\linewidth}\raggedleft
Std. Error
\end{minipage} & \begin{minipage}[b]{\linewidth}\raggedleft
t value
\end{minipage} & \begin{minipage}[b]{\linewidth}\raggedleft
Pr(\textgreater\textbar t\textbar)
\end{minipage} \\
\midrule()
\endfirsthead
\toprule()
\begin{minipage}[b]{\linewidth}\raggedright
\end{minipage} & \begin{minipage}[b]{\linewidth}\raggedleft
Estimate
\end{minipage} & \begin{minipage}[b]{\linewidth}\raggedleft
Std. Error
\end{minipage} & \begin{minipage}[b]{\linewidth}\raggedleft
t value
\end{minipage} & \begin{minipage}[b]{\linewidth}\raggedleft
Pr(\textgreater\textbar t\textbar)
\end{minipage} \\
\midrule()
\endhead
(Intercept) & 6804.8860355 & 9929.3177930 & 0.6853327 & 0.5029432 \\
LastWeek & 0.9820815 & 0.0144272 & 68.0714024 & 0.0000000 \\
\bottomrule()
\end{longtable}

\hypertarget{a}{%
\subsubsection*{a}\label{a}}
\addcontentsline{toc}{subsubsection}{a}

The confidence intervals for \(\beta_1\) are given by

\begin{Shaded}
\begin{Highlighting}[]
\FunctionTok{confint}\NormalTok{(pb\_fit1)[}\DecValTok{2}\NormalTok{, ]}
\end{Highlighting}
\end{Shaded}

\begin{verbatim}
##     2.5 %    97.5 % 
## 0.9514971 1.0126658
\end{verbatim}

As per the question, 1 seems like a plausible value given that returns
are likely to be similar from one week to another (although exactly 1 is
incredibly unlikely).

\hypertarget{b}{%
\subsubsection*{b}\label{b}}
\addcontentsline{toc}{subsubsection}{b}

We proceed to test the hypotheses \[
\begin{gather}
H_0:\beta_0 = 10000 \\
H_1:\beta_0 \neq 10000
\end{gather}
\]

by running

\begin{Shaded}
\begin{Highlighting}[]
\NormalTok{h\_0 }\OtherTok{\textless{}{-}} \DecValTok{10000}
\NormalTok{h\_obs }\OtherTok{\textless{}{-}} \FunctionTok{coef}\NormalTok{(pb\_fit1)[[}\DecValTok{1}\NormalTok{]]}
\NormalTok{h\_obs\_se }\OtherTok{\textless{}{-}} \FunctionTok{summary}\NormalTok{(pb\_fit1)}\SpecialCharTok{$}\NormalTok{coef[}\DecValTok{1}\NormalTok{, }\DecValTok{2}\NormalTok{]}
\NormalTok{tobs }\OtherTok{\textless{}{-}}\NormalTok{ (h\_obs }\SpecialCharTok{{-}}\NormalTok{ h\_0) }\SpecialCharTok{/}\NormalTok{ h\_obs\_se}
\NormalTok{(pobs }\OtherTok{\textless{}{-}} \DecValTok{2} \SpecialCharTok{*} \FunctionTok{pt}\NormalTok{(}\FunctionTok{abs}\NormalTok{(tobs), }\FunctionTok{nrow}\NormalTok{(playbill) }\SpecialCharTok{{-}} \DecValTok{2}\NormalTok{, }\AttributeTok{lower.tail =} \ConstantTok{FALSE}\NormalTok{))}
\end{Highlighting}
\end{Shaded}

\begin{verbatim}
## [1] 0.7517807
\end{verbatim}

which leads us to accept the null hypothesis, \(t(16) = -0.3217858\),
\(p = 0.7517807\).

\hypertarget{c}{%
\subsubsection*{c}\label{c}}
\addcontentsline{toc}{subsubsection}{c}

We make a prediction, including prediction interval, for a 400,000\$ box
office result in the previous week:

\begin{Shaded}
\begin{Highlighting}[]
\FunctionTok{predict}\NormalTok{(pb\_fit1, }\FunctionTok{data.frame}\NormalTok{(}\AttributeTok{LastWeek =} \DecValTok{400000}\NormalTok{), }\AttributeTok{interval =} \StringTok{"prediction"}\NormalTok{)}
\end{Highlighting}
\end{Shaded}

\begin{verbatim}
##        fit      lwr      upr
## 1 399637.5 359832.8 439442.2
\end{verbatim}

A prediction of 450,000\$ is \textbf{not} feasible, given it is far
outside our 95\% prediction interval.

\hypertarget{d}{%
\subsubsection*{d}\label{d}}
\addcontentsline{toc}{subsubsection}{d}

This seems like an okay rule given the almost-perfect correlation from
one week to another; however, looking at the residuals we see that there
are at least three values that are predicted badly (Figure
@ref(fig:pb-resid))

\begin{Shaded}
\begin{Highlighting}[]
\FunctionTok{par}\NormalTok{(}\AttributeTok{mfrow =} \FunctionTok{c}\NormalTok{(}\DecValTok{2}\NormalTok{, }\DecValTok{2}\NormalTok{))}
\FunctionTok{plot}\NormalTok{(pb\_fit1)}
\end{Highlighting}
\end{Shaded}

\begin{figure}
\centering
\includegraphics{hw1-sol_files/figure-latex/pb-resid-1.pdf}
\caption{Residuals for our linear fit to the playbill data.}
\end{figure}

\hypertarget{indicators}{%
\subsection{Indicators}\label{indicators}}

\begin{Shaded}
\begin{Highlighting}[]
\NormalTok{indicators }\OtherTok{\textless{}{-}} \FunctionTok{read.table}\NormalTok{(}\AttributeTok{file=}\StringTok{\textquotesingle{}C:/Users/tonyg/Desktop/Academic/Grad/HUDM 5126/indicators.txt\textquotesingle{}}\NormalTok{,}\AttributeTok{header =} \ConstantTok{TRUE}\NormalTok{)}
\end{Highlighting}
\end{Shaded}

We begin by fitting our linear model to the data (Table
@ref(tab:indicators-summary)).

\begin{Shaded}
\begin{Highlighting}[]
\NormalTok{ind\_fit1 }\OtherTok{\textless{}{-}} \FunctionTok{lm}\NormalTok{(PriceChange }\SpecialCharTok{\textasciitilde{}}\NormalTok{ LoanPaymentsOverdue, }\AttributeTok{data =}\NormalTok{ indicators)}
\FunctionTok{kable}\NormalTok{(}\FunctionTok{summary}\NormalTok{(ind\_fit1)}\SpecialCharTok{$}\NormalTok{coef,}
      \AttributeTok{booktabs =} \ConstantTok{TRUE}\NormalTok{,}
      \AttributeTok{caption =} \StringTok{"Coefficients for our linear model to the indicators data set."}\NormalTok{)}
\end{Highlighting}
\end{Shaded}

\begin{longtable}[]{@{}
  >{\raggedright\arraybackslash}p{(\columnwidth - 8\tabcolsep) * \real{0.2857}}
  >{\raggedleft\arraybackslash}p{(\columnwidth - 8\tabcolsep) * \real{0.1429}}
  >{\raggedleft\arraybackslash}p{(\columnwidth - 8\tabcolsep) * \real{0.1571}}
  >{\raggedleft\arraybackslash}p{(\columnwidth - 8\tabcolsep) * \real{0.1429}}
  >{\raggedleft\arraybackslash}p{(\columnwidth - 8\tabcolsep) * \real{0.2714}}@{}}
\caption{Coefficients for our linear model to the indicators data
set.}\tabularnewline
\toprule()
\begin{minipage}[b]{\linewidth}\raggedright
\end{minipage} & \begin{minipage}[b]{\linewidth}\raggedleft
Estimate
\end{minipage} & \begin{minipage}[b]{\linewidth}\raggedleft
Std. Error
\end{minipage} & \begin{minipage}[b]{\linewidth}\raggedleft
t value
\end{minipage} & \begin{minipage}[b]{\linewidth}\raggedleft
Pr(\textgreater\textbar t\textbar)
\end{minipage} \\
\midrule()
\endfirsthead
\toprule()
\begin{minipage}[b]{\linewidth}\raggedright
\end{minipage} & \begin{minipage}[b]{\linewidth}\raggedleft
Estimate
\end{minipage} & \begin{minipage}[b]{\linewidth}\raggedleft
Std. Error
\end{minipage} & \begin{minipage}[b]{\linewidth}\raggedleft
t value
\end{minipage} & \begin{minipage}[b]{\linewidth}\raggedleft
Pr(\textgreater\textbar t\textbar)
\end{minipage} \\
\midrule()
\endhead
(Intercept) & 4.514494 & 3.3240193 & 1.358143 & 0.1932624 \\
LoanPaymentsOverdue & -2.248520 & 0.9033113 & -2.489197 & 0.0241941 \\
\bottomrule()
\end{longtable}

\hypertarget{a-1}{%
\subsubsection*{a}\label{a-1}}
\addcontentsline{toc}{subsubsection}{a}

The 95\% confidence interval for the \(\beta_1\) estimate is, as before:

\begin{Shaded}
\begin{Highlighting}[]
\FunctionTok{confint}\NormalTok{(ind\_fit1)[}\DecValTok{2}\NormalTok{, ]}
\end{Highlighting}
\end{Shaded}

\begin{verbatim}
##      2.5 %     97.5 % 
## -4.1634543 -0.3335853
\end{verbatim}

There is reason to believe that there is a negative trend.

\hypertarget{b-1}{%
\subsubsection*{b}\label{b-1}}
\addcontentsline{toc}{subsubsection}{b}

We now create a confidence interval for \(\text{E}[Y|X=4]\):

\begin{Shaded}
\begin{Highlighting}[]
\FunctionTok{predict}\NormalTok{(ind\_fit1, }\FunctionTok{data.frame}\NormalTok{(}\AttributeTok{LoanPaymentsOverdue =} \DecValTok{4}\NormalTok{), }\AttributeTok{interval =} \StringTok{"confidence"}\NormalTok{)}
\end{Highlighting}
\end{Shaded}

\begin{verbatim}
##         fit       lwr       upr
## 1 -4.479585 -6.648849 -2.310322
\end{verbatim}

0\% is not a reasonable estimate for \(\text{E}[Y|X=4]\) since the 95\%
confidence limit is far below 0.

\hypertarget{invoices}{%
\subsection{Invoices}\label{invoices}}

\hypertarget{a-2}{%
\subsubsection*{a}\label{a-2}}
\addcontentsline{toc}{subsubsection}{a}

We first find a 95\% confidence level for \(\beta_0\) using the output
printed in the book.

\begin{Shaded}
\begin{Highlighting}[]
\NormalTok{beta0 }\OtherTok{\textless{}{-}} \FloatTok{0.6417099}
\NormalTok{beta0\_se }\OtherTok{\textless{}{-}} \FloatTok{0.122707}
\NormalTok{beta0\_t }\OtherTok{\textless{}{-}} \FloatTok{5.248}
\NormalTok{beta0\_margin }\OtherTok{\textless{}{-}} \FloatTok{1.96} \SpecialCharTok{*}\NormalTok{ beta0\_se}
\NormalTok{(beta0\_95 }\OtherTok{\textless{}{-}} \FunctionTok{c}\NormalTok{(beta0 }\SpecialCharTok{{-}}\NormalTok{ beta0\_margin, beta0 }\SpecialCharTok{+}\NormalTok{ beta0\_margin))}
\end{Highlighting}
\end{Shaded}

\begin{verbatim}
## [1] 0.4012042 0.8822156
\end{verbatim}

Thus, the confidence limit it 0.4012042, 0.8822156

\hypertarget{b-2}{%
\subsubsection*{b}\label{b-2}}
\addcontentsline{toc}{subsubsection}{b}

We have the two-sided hypotheses \[\begin{gathered}
H_0: \beta_1 = 0.01\\
H_1: \beta_1 \neq 0.01.
\end{gathered}\]

\begin{Shaded}
\begin{Highlighting}[]
\NormalTok{beta }\OtherTok{\textless{}{-}} \FloatTok{0.01}
\NormalTok{beta\_obs\_se }\OtherTok{\textless{}{-}} \FloatTok{0.0008184}
\NormalTok{beta\_obs }\OtherTok{\textless{}{-}} \FloatTok{0.0112916}
\NormalTok{tval }\OtherTok{\textless{}{-}}\NormalTok{ (beta }\SpecialCharTok{{-}}\NormalTok{ beta\_obs) }\SpecialCharTok{/}\NormalTok{ beta\_obs\_se}
\NormalTok{(pobs }\OtherTok{\textless{}{-}} \DecValTok{2} \SpecialCharTok{*} \FunctionTok{pt}\NormalTok{(}\FunctionTok{abs}\NormalTok{(tval), }\DecValTok{30} \SpecialCharTok{{-}} \DecValTok{1}\NormalTok{, }\AttributeTok{lower.tail =} \ConstantTok{FALSE}\NormalTok{))}
\end{Highlighting}
\end{Shaded}

\begin{verbatim}
## [1] 0.1253666
\end{verbatim}

We fail to reject the null hypothesis, \(t(29) = -1.5782014\),
\(p=0.1253666\). We cannot say that the true average processing time is
significantly different from 0.01 hours.

\hypertarget{c-1}{%
\subsubsection*{c}\label{c-1}}
\addcontentsline{toc}{subsubsection}{c}

From the exercise description we have the expected value \[
\text{Time} = 0.6417099+\text{Invoices}\times 0.0112916
\] for the series. Next, we'll predict the processing time for 130
invoices using the output given in the exercise.

\begin{Shaded}
\begin{Highlighting}[]
\NormalTok{beta0 }\OtherTok{\textless{}{-}} \FloatTok{0.6417099}
\NormalTok{beta1 }\OtherTok{\textless{}{-}} \FloatTok{0.0112916}
\NormalTok{rse }\OtherTok{\textless{}{-}} \FloatTok{0.3298}
\NormalTok{n }\OtherTok{\textless{}{-}} \DecValTok{30}
\NormalTok{df }\OtherTok{\textless{}{-}}\NormalTok{ n }\SpecialCharTok{{-}} \DecValTok{2}
\NormalTok{rss }\OtherTok{\textless{}{-}}\NormalTok{ rse}\SpecialCharTok{\^{}}\DecValTok{2} \SpecialCharTok{*}\NormalTok{ df}
\NormalTok{mse }\OtherTok{\textless{}{-}}\NormalTok{ rss }\SpecialCharTok{/}\NormalTok{ n }
\NormalTok{time }\OtherTok{\textless{}{-}}\NormalTok{ beta0 }\SpecialCharTok{+}\NormalTok{ beta1 }\SpecialCharTok{*} \DecValTok{130}
\NormalTok{err }\OtherTok{\textless{}{-}} \FunctionTok{qt}\NormalTok{(}\FloatTok{0.975}\NormalTok{, }\DecValTok{28}\NormalTok{) }\SpecialCharTok{*} \FunctionTok{sqrt}\NormalTok{(mse) }\SpecialCharTok{*} \FunctionTok{sqrt}\NormalTok{(}\DecValTok{1} \SpecialCharTok{+} \DecValTok{1} \SpecialCharTok{/}\NormalTok{ n) }\CommentTok{\# since x0 = xbar}
\NormalTok{upr }\OtherTok{\textless{}{-}}\NormalTok{ time }\SpecialCharTok{+}\NormalTok{ err}
\NormalTok{lwr }\OtherTok{\textless{}{-}}\NormalTok{ time }\SpecialCharTok{{-}}\NormalTok{ err}
\end{Highlighting}
\end{Shaded}

Which results in a point estimate of 2.1096179, 95\% CI:
\([1.446172, 2.7730638]\).

\end{document}
